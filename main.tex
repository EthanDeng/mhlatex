% !TEX options=--shell-escape
%!TEX program = xelatex
\documentclass[11pt,a4paper]{dlove}

\title{\bfseries 基本无害的经济学 \LaTeX{} 技巧}
\author{\href{https://ddswhu.me/}{\bfseries 邓 东 升} \\  Elegant\LaTeX{} 项目组}
\date{2019 年 10 月 18 日}
\newcommand{\heart}{\textcolor{red}{\ensuremath\heartsuit}}
\usepackage{amsmath}
\setminted[tex]{breaklines,breaksymbolleft={}}
\begin{document}

\maketitle

此文档为经济学专业的 \LaTeX{} 技巧总结,包括环境搭建、基础知识、参考文献以及幻灯片制作等内容,仅作为经济学专业的师生作为入门 \LaTeX{} 使用。使用 \mintinline{tex}{dlove} \heart{} 模板和 \hologo{XeLaTeX} 编译完成。

\section{环境搭建}
为了使用 \LaTeX{},推荐安装 \LaTeX{} 套装,目前有的套装有
\begin{itemize}
\item C\TeX{}:\hologo{MiKTeX} 和 编辑器 WinEdt,Win 系统,已过时;
\item \TeX{} Live:内核 texlive,编辑器 \TeX{}works,跨系统;
\item Mac\TeX{}:内核,Mac 系统;
\end{itemize} 

推荐使用 \TeX{} Live 2019,可以使用 \TeX{}works 或者配合其他编辑器(比如 Sublime Text、Visual Studio Code)的插件进行编写,后面我们会细说这部分。

\subsection{安装 \TeX{} Live}
首先,我们进入 \TeX{} Live 的\href{https://www.tug.org/texlive/}{官网}地址,点击页面中的 \href{https://www.tug.org/texlive/acquire-netinstall.html}{download},然后可以选择在线安装或者下载安装文件之后离线安装,推荐使用离线安装。
\begin{itemize}
	\item \textbf{在线安装}:点击 \href{http://mirror.ctan.org/systems/texlive/tlnet/install-tl-windows.exe}{install-tl-windows.exe}(Windows) 或者 \href{http://mirror.ctan.org/systems/texlive/tlnet/install-tl-unx.tar.gz}{install-tl-unix.tar.gz}(Linux/Unix),视自己系统选择,然后按照提示进行安装。
	\item \textbf{离线安装}:首先下载镜像文件,点击 \href{https://ctan.org/mirrors}{generic mirror.ctan.org url},这个时候我们会跳转到 \TeX{} Live 的\href{https://ctan.org/mirrors}{镜像站},下拉找到国内的镜像站。在国内的镜像列表中,选择离自己比较近的地区的镜像进行下载\footnote{比如上海的用户可以选择\href{https://mirrors.sjtug.sjtu.edu.cn/ctan/}{上海交大的镜像地址},然后往上拉找到 \TeX{} Live,点击进入上海交大 \TeX{} Live 的\href{https://mirrors.sjtug.sjtu.edu.cn/ctan/systems/texlive/}{下载地址},选择 \href{https://mirrors.sjtug.sjtu.edu.cn/ctan/systems/texlive/Images/}{./images/},然后将 \href{https://mirrors.sjtug.sjtu.edu.cn/ctan/systems/texlive/Images/texlive.iso}{texlive.iso}下载即可。}。下载镜像文件之后,使用资源管理器或者镜像挂载工具\footnote{推荐使用 \href{http://wincdemu.sysprogs.org/}{WinCDEmu} 进行挂载,WinCDEmu 下载后直接安装即可,这里不再赘述。}进行挂载,\textsf{\textcolor{FireBrick}{请不要使用压缩软件对镜像文件解压缩}}。
\end{itemize}

\subsection{配置编译环境}

\section{基础知识}

\subsection{最简示例}
\begin{minted}{tex}
\documentclass{article}

\begin{document}

Hello World.

\end{document}
\end{minted}

\subsection{中文支持}
目前流行的中文支持有两个方式:
\begin{itemize}
\item \mintinline{tex}{ctex} 宏包,或者与其相适应的 \mintinline{tex}{ctexart} 等文类。
\item \mintinline{tex}{xeCJK} 宏包,需要使用 \hologo{XeLaTeX} 编译。
\end{itemize}

\subsection{数学字母}
常用的一些希腊字母见表~\ref{tab:greek_letters},需要注意的是这些希腊字母需要在数学模式(比如 \mintinline{tex}{$\alpha$})或者数学环境中使用。
\begin{table}[htbp]
  \centering
  \small
  \caption{希腊字母表}
    \begin{tabular}{llllll}
    \toprule
    \textsf{符号} & \textsf{命令} & \textsf{符号} & \textsf{命令} & \textsf{符号} & \textsf{命令} \\
    \midrule
    $\alpha$ & \mintinline{tex}{\alpha} & $\iota$ & \mintinline{tex}{\iota} & $\sigma$\; $\Sigma$ & \mintinline{tex}{\sigma \Sigma} \\
    $\beta$ & \mintinline{tex}{\beta} & $\kappa$ & \mintinline{tex}{\kappa} & $\tau$ & \mintinline{tex}{\tau} \\
    $\gamma$\; $\Gamma$ & \mintinline{tex}{\gamma \Gamma} & $\lambda$ \; $\Lambda$ & \mintinline{tex}{\lambda \Lambda} & $\upsilon$\; $\Upsilon$ & \mintinline{tex}{\upsilon \Upsilon} \\
    $\delta$ \; $\Delta$ & \mintinline{tex}{\delta \Delta} & $\mu$ & \mintinline{tex}{\mu} & $\phi$\; $\Phi$ & \mintinline{tex}{\phi \Phi} \\
    $\epsilon$ & \mintinline{tex}{\epsilon} & $\nu$ & \mintinline{tex}{\nu} & $\chi$ & \mintinline{tex}{\chi} \\
    $\zeta$ & \mintinline{tex}{\zeta} &  $\pi$ \; $\Pi$& \mintinline{tex}{\pi \Pi} & $\psi$\; $\Psi$ & \mintinline{tex}{\psi \Psi} \\
    $\eta$ & \mintinline{tex}{\eta} & $\rho$ & \mintinline{tex}{\rho}  & $\omega$\; $\Omega$ & \mintinline{tex}{\omega \Omega} \\
    $\theta$ & \mintinline{tex}{\theta} & $\varepsilon$ &  \mintinline{tex}{\varepsilon} &       &  \\
    \bottomrule
    \end{tabular}%
  \label{tab:greek_letters}%
\end{table}%

\subsection{文本模式与数学模式}
在 \LaTeX{} 中,文本和数学是作为两个独立的不同模式存在的,如果需要在文本模式中输入数学式,需要使用英文状态下的美元符号 \mintinline{tex}{$} 将数学命令包围,比如 \mintinline{tex}{$\alpha$} 输出为 $\alpha$。

\begin{code}
假设 $y_{i}$ 是被解释变量的第 $i$ 次观测,$x_{i}$ 是解释变量的第 $i$ 次观测,设定回归方程为 $y_{i} = \alpha + \beta x_{i} + \varepsilon_{i}$。
\end{code}
\begin{preview}
假设 $y_{i}$ 是被解释变量的第 $i$ 次观测,$x_{i}$ 是解释变量的第 $i$ 次观测,设定回归方程为 $y_{i} = \alpha + \beta x_{i} + \varepsilon_{i}$。
\end{preview}

\subsection{数学环境}
数学环境中,最简单的就是 \mintinline{tex}{equation} 环境,这个环境会对数学公式进行自动编号,如果不需要编号,可以使用 \mintinline{tex}{equation*} 环境。\par

\begin{code}
\begin{equation}
y_{i} = \alpha + \beta x_{i} + \varepsilon_{i}
\end{equation}
\end{code}
\begin{preview}
\begin{equation}
y_{i} = \alpha + \beta x_{i} + \varepsilon_{i}
\end{equation}
\end{preview}

\section{文献}

\section{幻灯片}
Beamer 是 \LaTeX{} 用于制作幻灯片的一个文类,由于它的格式简洁、易于使用、方便展示数学公式和逻辑演绎,在学术界特别是国外非常受欢迎。hao下面分别是是英文 Beamer 和中文 Beamer 的一个简单示例:

\begin{code}
\documentclass{beamer}

% title information
\title{An Example of Beamer Class}
\author{Dongsheng DENG}
\institute{Fudan University}
\date{\today}

\begin{document}
\maketitle

\begin{frame}{frame title}
Be honest rather clever.
\end{frame}

\end{document}
\end{code}
\begin{code}
\documentclass{beamer}
\usepackage[UTF8,scheme=plain]{ctex}
% 标题信息
\title{Beamer 文类示例}
\author{邓东升}
\institute{复旦大学}
\date{2019 年 10 月 23 日}

\begin{document}
\maketitle

\begin{frame}{帧标题}
有志者事竟成,百二秦关终属楚。
\end{frame}

\end{document}
\end{code}

\section{文档说明}

本文档使用了 \mintinline{tex}{fontspec} 和  \mintinline{tex}{xeCJK} 设置英文字体和中文字体,用户需要的字体列表如下:
\begin{table}[htbp]
  \centering
  \caption{本文档字体设置}
    \begin{tabular}{lccc}
    \toprule
          & \textbf{衬线字体}  & \textbf{非衬线字体} & \textbf{等宽字体}   \\
    \midrule
    英文/\mintinline{tex}{fontspec}    & Amiri & \textsf{Roboto} & \texttt{Ubuntu Mono} \\
    中文/\mintinline{tex}{xeCJK}   & 方正书宋简体 & \textsf{方正楷体简体} & \texttt{方正黑体简体}  \\
    \bottomrule
    \end{tabular}%
  \label{tab:fontset}%
\end{table}%

需要注意的是,在 Win 10 中,安装字体时需要\textsf{\textcolor{FireBrick}{为所有用户安装}},否则即便安装了字体,\LaTeX{} 也无法找到。

另外,本文高亮使用了 \mintinline{tex}{minted} 宏包,所以,需要调用 \mintinline{shell}{-shell-escape} 选项并用 \hologo{XeLaTeX} 进行编译,如果使用命令行编译,命令如下:
\begin{minted}{shell}
xelatex --shell-escape main.tex
\end{minted}

\end{document}